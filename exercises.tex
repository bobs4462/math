\documentclass[a4paper, 11pt]{article}
\usepackage[margin=1.5cm]{geometry}
\usepackage{amssymb}
\renewcommand{\familydefault}{\ttdefault}
\setlength\parindent{0pt}

\begin{document}
\section*{Axioms}
\subsection*{Field Axioms}

\begin{enumerate}
  \item COMMUTATIVE LAWS: $x + y = y + x$ and $xy = yx$
  \item ASSOCIATIVE LAWS: $x + (y + z) = (x + y) + z$ and $x(yz) = (xy)z$
  \item DISTRIBUTIVE LAW: $x(y + z) = xy + xz$
  \item EXISTENCE OF IDENTITY ELEMENTS: there exists two distinct real numbers: 0 and 1,\\ such that $x + 0 = x$ and $1x = x$
  \item EXISTENCE OF NEGATIVES: for $x \in \mathbb{R}$ there is $y \in \mathbb{R}$ such that $x + y = 0$
  \item EXISTENCE OF RECIPROCALS for $x \in \mathbb{R} \ | \ x \ne 0$ there is $y \in \mathbb{R}$ such that $xy = 1$
\end{enumerate}

\subsection*{Order Axioms}
\begin{enumerate}
  \item If $x, y \in \mathbb{R}^+$, so are $x + y$ and $xy$
  \item $\forall x \in \mathbb{R} \ | \ x \ne 0$ either $x \in \mathbb{R}^+$ or $-x \in \mathbb{R}^+$ 
  \item $0 \notin \mathbb{R}^+$
\end{enumerate}

\subsection*{Least uppper bound Axiom}
\begin{enumerate}
  \item $\forall S: S \ne \emptyset, S \subset \mathbb{R}$, if S is bounded above, then S has a \emph{supremum}, i.e. $\exists B: B = sup\ S$
\end{enumerate}


\hrulefill

\section*{Exercises 1.3.12}

\hrulefill

1. If $x \in \mathbb{R}$ and $y \in \mathbb{R}$ with $x < y$, prove that there is at least one $z \in \mathbb{R}$, so that it satisfies $x < z < y$.

\emph{Proof:} If $x < y$ then $y - x > 0$ adding $y - x$ to both sides of inequality, we have:

$$2(y - x) > y - x \ \textrm{or} \ y - x > \frac{y - x}{2}$$

because $y - x > 0$ and $\frac{1}{2} > 0$, then 
$$\frac{y - x}{2} > 0$$

Hence, adding x to inequality:

$$0 < \frac{y - x}{2} < y - x \Rightarrow x < \frac{y - x}{2} < y$$

setting $z = \frac{y - x}{2}$ we prove $x < z < y$

\hrulefill

2. If $x \in R$, prove that there are $m, n \in \mathbb{Z}$ such that $m < x < n$.

\emph{Proof:} By theorem 1.32 we have there exists $n \in \mathbb{Z}$ such that
$n > x$, also by the same way we can prove that there exists $m \in \mathbb{Z}$
such that $m < x$, because if it didn't exist then $x$ would be the lower bound
for $\mathbb{Z}$

\hrulefill

3. If $x > 0$, prove that there are $n \in \mathbb{Z}^+$ such that $n^{-1} < x$.

\emph{Proof:} By theorem 1.30 we have: if $x > 0$ then $\forall y \in \mathbb{R}, \exists n : nx > y$,
letting $y = 1$ we have $nx > 1$ and $x > \frac{1}{n}$

\hrulefill

\emph{BREAKING HERE. APOSTOL IS KINDA TOO HARD ON PROOFS. MIGHT RETURN TO IT LATER}

\section*{SPIVAK ONE VARIABLE CALCULUS}
\section*{CHAPTER 1}
\subsection*{Properties of numbers}
\begin{enumerate}
  \item $\forall (a, b, c),  a + (b + c) = (a + b) + c$
  \item $\forall a,  a + 0 = 0 + a = a$
  \item $\forall a, a + (-a) = 0$
  \item $\forall (a, b), a + b = b + a$
  \item $\forall (a, b, c), a(bc) = (ab)c$
  \item $\forall a, a1 = 1a = a$ and $1 \ne 0$
  \item $\forall a: a \ne 0, \exists a^{-1}$ such that $aa^{-1} = 1$
  \item $\forall (a, b), ab = ba$
  \item $\forall (a, b, c), a(b + c) = ab + ac$
  \item $\forall a, a = 0$ or $a \in \mathbb{R}^+$ or $-a \in \mathbb{R}^+$
  \item $\forall (a, b)$ if $a \in \mathbb{R}^+$ and $b \in \mathbb{R}^+$ then $a + b \in \mathbb{R}^+$
  \item $\forall (a, b)$ if $a \in \mathbb{R}^+$ and $b \in \mathbb{R}^+$ then $ab \in \mathbb{R}^+$
\end{enumerate}

\subsection*{Theorems related to numbers}
\begin{enumerate}
  \item $\forall (a, b), |a + b| \leq |a| + |b|$
\end{enumerate}

\subsection*{\underline{Exercises}}
1.1 IF $ax = a$ for some $a \ne 0$, then $x = 1$\\
\emph{Proof:} 
By multiplying both sides of equation by reciprocal of $a$ we have:
$$aa^{-1}x = aa^{-1} \Rightarrow 1x = 1 \ [P7]$$
applying P6 we get $x = 1$

\hrulefill

1.2 $x^2 - y^2 = (x - y)(x + y)$\\
\emph{Proof:}
Apply P9 twice, let $(x + y) = c$ then 
$$(x - y)c = xc - yc$$
substitute back $c = (x + y)$ and reapply P9
$$x(x + y) - y(x + y) = xx + xy - yx - yy$$
using P8 and P3 we have:
$$xx + xy - yx - yy = xx + xy - xy - yy = xx - yy = x^2 - y^2$$

\hrulefill

1.3 $x^2 = y^2 \Rightarrow x = y \ or \ x = -y$\\
\emph{Proof:}
We can observe the fact that:
$$\sqrt{x^2} = |x|$$
then we can take square roots of both sides of equation:
$$\sqrt{x^2} = \sqrt{y^2} = |x| = |y|$$
then we have 4 cases:
$$x \geq 0, y \geq 0$$
$$x \leq 0, y \geq 0$$
$$x \leq 0, y \leq 0$$
$$x \geq 0, y \leq 0$$
if the 1st or 3rd case is true, then $x = y$, if 2nd or 4th case is true, then $x = - y$ 

\hrulefill

1.4 $x^3 - y^3 = (x - y)(x^2 + xy + y^2)$\\
\emph{Proof:}
Triple application of P9:
$$(x - y)(x^2 + xy + y^2) = x^3 + x^2y + xy^2 - x^2y - xy^2 - y^3$$
using P8 and P3 again we have:
$$x^3 + x^2y + xy^2 - x^2y - xy^2 - y^3 = x^3 + x^2y - x^2y + xy^2 - xy^2 - y^3 = x^3 - y^3$$ 

\hrulefill

1.5 $x^n - y^n = (x - y)(x^{n-1} + x^{n-2}y + x^{n-3}y^2 + \ldots + xy^{n-2} + y^{n-1})$\\
\emph{Proof:} 
We can rewrite the right hand side as as
$$(x - y) \sum_{i=0}^{n-1}x^iy^{n-1-i} = x\sum_{i=0}^{n-1}x^iy^{n-1-i} - y\sum_{i=0}^{n-1}x^iy^{n-1-i} = $$
$$x^n + \sum_{i=0}^{n-2}x^{i+1}y^{n-1-i} - (y^n + \sum_{i=1}^{n-1}x^iy^{n-i}) = $$
from here, we should be able to prove that
$$\sum_{i=0}^{n-2}x^{i+1}y^{n-1-i} = \sum_{i=1}^{n-1}x^iy^{n-i}$$
if we were to substitute $i - 1$ with $k$ then
$$\sum_{i=0}^{n-2}x^{i+1}y^{n-1-i} = \sum_{k=0}^{n-2}x^{k+1}y^{n-(k+1)} = \sum_{k=0}^{n-2}x^{k+1}y^{n-1-k}$$
then, it's possible to rewrite our original equation as
$$x^n - y^n + (\sum_{i=0}^{n-2}x^{i+1}y^{n-1-i} - \sum_{i=1}^{n-1}x^iy^{n-i} ) = x^n - y^n + 0$$

\hrulefill

1.6 $x^3 + y^3 = (x + y)(x^2 - xy + y^2)$\\
\emph{Proof:} 
Using proof of in 1.4 we can replace $y$ with $-y$ and as a result we have
$$(x - (-y))(x^2 + x(-y) + y^2) = x^3 + x^2(-y) + xy^2 - x^2(-y) - xy^2 - (-y)^3$$
which can be simplified as
$$(x + y)(x^2 - xy + y^2) = x^3 - x^2y + xy^2 + x^2y - xy^2 + y^3 = x^3 + y^3$$

\hrulefill

2 what's wrong with the following proof? Let $x = y$, then
$$xy = x^2$$
$$x^2 - y^2 = xy - y^2$$
$$(x - y)(x + y) = y(x - y)$$
$$(x + y) = y$$
$$2y = y$$
$$2 = 1$$
\emph{Solution:}
At step 3, both sides of equation was divided by $(x - y)$ as a common factor.\\
But from initial conditions $x = y$, it follows, that $x - y = 0$, so what
actually happened, is division by zero (which is undefined), that's why the answer is nonsense.

\hrulefill

3.1 $\frac{a}{b} = \frac{ac}{bc}$, where $b, c \ne 0$\\
\emph{Proof:}
Let's multiply the left hand side of the equation by number $c \ne 0$ and it's\\
multiplicative inverse $c^{-1}$, and according to P7 $cc^{-1} = 1$, and according to P6 $a1 = a$.
$$\frac{a}{b}1 = \frac{a}{b}cc^{-1} = acb^{-1}c^{-1}$$
Which can be rewritten as
$$acb^{-1}c^{-1} = ac(bc)^{-1} = \frac{ac}{bc}$$

\hrulefill

3.2 $\frac{a}{b} + \frac{c}{d} = \frac{ad + cb}{bd}$, where $b, d \ne 0$\\
\emph{Proof:}
Using the same technique as in 3.1 multiply lhs of equation by $dd^{-1}$ and $bb^{-1}$
$$(\frac{a}{b} + \frac{c}{d})dd^{-1}bb^{-1} = ab^{-1}bb^{-1}dd^{-1} + cd^{-1}bb^{-1}dd^{-1} = $$
$$(ad)(b^{-1}d^{-1}) + cb(b^{-1}d^{-1}) = (ad + cb)(b^{-1}d^{-1}) = \frac{ad + cb}{bd}$$

\hrulefill

3.5 $\frac{a}{b} / \frac{c}{d} = \frac{ad}{bc}$, where $b, d, c \ne 0$\\
\emph{Proof:}
Let's rewrite it in a different form
$$ab^{-1}(cd^{-1})^{-1} = ab^{-1}(c^{-1}(d^{-1})^{-1})$$
because $(d^{-1})^{-1} = d$, then we can make final rearrangement
$$ab^{-1}(c^{-1}(d^{-1})^{-1}) = adb^{-1}c^{-1} = \frac{ad}{bc}$$

\hrulefill

4.4 find $x$ such that $(x - 1)(x - 3) > 0$\\
\emph{Solution:}
For a product of two numbers to be positive, they both need to be positive or\\
both should be negative. Thus:
$$x - 1 > 0, x - 3 > 0$$ or $$x - 1 < 0, x - 3 < 0$$ 
solving those inequalities, we have:
$$x > 1, x > 3 \Rightarrow x > 3$$ or $$x < 1, x < 3 \Rightarrow x < 1$$
i.e $x > 3, x < 1$

\hrulefill

4.7 Find $x$ such that $x^2 - x + 10 > 16$\\
\emph{Solution:}
Let's try to solve quadratic equation of the form $x^2 - x - 6 = 0$, and if it
does not have roots, then $\forall x \in \mathbb{R}$ the
inequality will hold true, if does have roots, then it's possible to factor
the lhs and try to solve inequality by parts
$$x^2 - x - 6 = 0$$
$$x = \frac{1 \pm \sqrt{1 + 24}}{2} = (-2, 3)$$
Hence the factore version of original inequality has the form:
$$(x + 2)(x - 3) > 0$$
which has the solution of $x > 3, x < -2$

\hrulefill

4.11 Find $x$ such that $2^x < 8$\\
\emph{Solution:} 
Lets take $\log_2$ of both sides of inequality:
$$\log_2{2^x} < \log_2{8} \Rightarrow x < 3$$

\hrulefill

4.14 Find $x$ such that $\frac{x - 1}{x + 1} > 0$\\
\emph{Solution:}
Let's rewrite the inequality:
$$(x - 1)(x + 1)^{-1} > 0$$
so, either $x - 1 > 0, (x + 1)^{-1} > 0$ or $x - 1 < 0, (x + 1)^{-1} < 0$, which resolves to
$$x > 1, (x + 1)^{-1} > 0 \Rightarrow (x + 1) > 0 \Rightarrow x > - 1$$
$$x < 1, (x + 1)^{-1} < 0 \Rightarrow (x + 1) < 0 \Rightarrow x < - 1$$
i.e the final answer is $x > 1, x < -1$

\hrulefill

5.1 $a < b, c < d \Rightarrow a + c < b + d$\\
\emph{Solution:}
$$a < b \Rightarrow b - a > 0$$
$$c < d \Rightarrow d - c > 0$$
adding those inequalities:
$$b - a + d - c > 0 \Rightarrow (b + d) - (a + c) > 0 \Rightarrow a + c < b + d$$

\hrulefill

5.2 $a < b \Rightarrow -a < -b$\\
\emph{Solution:}
$$a < b \Rightarrow b - a > 0 \Rightarrow -(-(b - a)) > 0 \Rightarrow -a - (-b) > 0 \Rightarrow -b < -a$$

\hrulefill

5.4 $a < b, c > 0 \Rightarrow  ac < bc$\\
\emph{Solution:}
$$a < b \Rightarrow b - a > 0$$
multiplying both sides by c:
$$c(b - a) > 0 \Rightarrow bc - ac > 0 \Rightarrow ac < bc$$

\hrulefill

5.7 $0 < a < 1 \Rightarrow a^2 > a $\\
\emph{Solution:}
$$a < 1 \Rightarrow 1 - a > 0$$
multiplying both sides by a:
$$a(1 - a) > 0 \Rightarrow 1a - aa > 0 \Rightarrow a^2 < a$$

\hrulefill

6 IF $0 < a < b$, then
$$a < \sqrt{ab} < \frac{a + b}{2} < b$$
\emph{Solution:}

\hrulefill

12.4 $|x - y| \leq |x| + |y|$\\
\emph{Proof:}
$$|x - y| = |x + (-y)| \leq |x| + |-y| = |x| + |y|$$

\hrulefill

17.2 If $b^2 - 4c < 0$ show that $\nexists x \in \mathbb{R}$ such that $x^2 + bx + c = 0$\\
\emph{Proof:}
Note that $b^2 - 4c < 0 \Rightarrow 4c - b^2 > 0 \Rightarrow c - b^2/4$. Then
$$x^2 + bx + c = 0 \Rightarrow (x + \frac{b}{2})^2 + (c - \frac{b^2}{4})$$
$$(x + \frac{b}{2})^2 + (c - \frac{b^2}{4}) \geq (c - \frac{b^2}{4}) > 0$$
Thus $(x + \frac{b}{2})^2 + (c - \frac{b^2}{4}) = x^2 + bx + c > 0, \forall x \in \mathbb{R}$ 

\hrulefill

\underline{\emph{Shwartz inequality}}:
$$x_1y_1 + x_2y_2 \leq \sqrt{x_1^2 + x_2^2}\sqrt{y_1^2 + y_2^2}$$
18.1 \underline{Proof 1}
$$(x_1^2 + x_2^2)(y_1^2 + y_2^2) = (x_1y_1 + x_2y_2)^2 + (x_1y_2 - x_2y_1)^2$$
$$(x_1^2 + x_2^2)(y_1^2 + y_2^2) = x_1^2y_1^2 + x_1^2y_2^2 + x_2^2y_1^2 + x_2^2y_2^2$$
$$(x_1^2y_1^2 + 2x_2y_2 + x_2^2y_2^2) + (x_1^2y_2^2 - 2x_2y_2 + x_2^2y_1^2) = (x_1y_1 + x_2y_2)^2 + (x_1y_2 - x_2y_1)^2$$
if we take squares of both sides
$$(x_1y_1 + x_2y_2)^2 \leq (\sqrt{x_1^2 + x_2^2}\sqrt{y_1^2 + y_2^2})^2 \Rightarrow$$
$$(x_1y_1 + x_2y_2)^2 \leq (x_1^2 + x_2^2)(y_1^2 + y_2^2) = (x_1y_1 + x_2y_2)^2 + (x_1y_2 - x_2y_1)^2$$
because $(x_1y_2 - x_2y_1)^2 \geq 0, \forall x \in \mathbb{R}$ 

\hrulefill

18.2 \underline{Proof 2}

\end{document}



